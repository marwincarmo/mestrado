\clearpage

\begin{appendix}
\section{Crenças e Atitudes Disfuncionais sobre o Sono (CADS-16)}
\label{cads-16}

Uma série de afirmações refletindo as crenças e atitudes das pessoas em relação ao sono estão listadas abaixo. Por favor, indique o quanto você concorda ou discorda de cada afirmação. Não há respostas certas ou erradas. Para cada afirmação, circule o número que corresponde à sua \underline{crença pessoal}. Por favor, responda todos as afirmações, mesmo que não se apliquem diretamente à sua situação.
\smallskip

\begin{flushleft}

{\setstretch{1.0}
\begin{center}
Discordo \hfill Concordo\\
Fortemente \hfill Fortemente
\end{center}

\medskip
\hrule
\medskip

\hspace*{6mm}\makebox[\linewidth][s]{0 1 2 3 4 5 6 \textcircled{7} 8 9 10 } \par

\bigskip
1. Preciso de 8 horas de sono para me sentir revigorado(a) e funcionar bem durante o dia.
\medskip
\hrule
\medskip
\hspace*{6mm}\makebox[\linewidth][s]{0 1 2 3 4 5 6 7 8 9 10 } \par


\bigskip
2. Quando não durmo o suficiente à noite, preciso recuperar o sono no dia seguinte com um cochilo ou dormindo mais na próxima noite.
\medskip
\hrule
\medskip
\hspace*{6mm}\makebox[\linewidth][s]{0 1 2 3 4 5 6 7 8 9 10 } \par

\bigskip
3. Estou preocupado(a) que a insônia crônica possa trazer consequências graves em minha saúde física. 
\medskip
\hrule
\medskip
\hspace*{6mm}\makebox[\linewidth][s]{0 1 2 3 4 5 6 7 8 9 10 } \par

\bigskip
4. Estou preocupado(a) que eu talvez perca o controle sobre minha habilidade de dormir.
\medskip
\hrule
\medskip
\hspace*{6mm}\makebox[\linewidth][s]{0 1 2 3 4 5 6 7 8 9 10 } \par

\bigskip
5. Sei que uma noite de sono ruim vai interferir nas minhas atividades cotidianas no dia seguinte.
\medskip
\hrule
\medskip
\hspace*{6mm}\makebox[\linewidth][s]{0 1 2 3 4 5 6 7 8 9 10 } \par

\bigskip
6. Para estar alerta e funcionar bem durante o dia, eu acredito que seria melhor tomar um remédio para dormir do que ter uma noite de sono ruim.
\medskip
\hrule
\medskip
\hspace*{6mm}\makebox[\linewidth][s]{0 1 2 3 4 5 6 7 8 9 10 } \par

\bigskip
7. Quando me sinto irritado(a), deprimido(a), ou ansioso(a) durante o dia, provavelmente foi porque não dormi bem na noite anterior. 
\medskip
\hrule
\medskip
\hspace*{6mm}\makebox[\linewidth][s]{0 1 2 3 4 5 6 7 8 9 10 } \par

\bigskip
8. Quando durmo mal uma noite, sei que irá atrapalhar meu sono pelo resto da semana. 
\medskip
\hrule
\medskip
\hspace*{6mm}\makebox[\linewidth][s]{0 1 2 3 4 5 6 7 8 9 10 } \par

\bigskip
9.	Sem uma noite de sono adequada, eu mal consigo funcionar no dia seguinte. 
\medskip
\hrule
\medskip
\hspace*{6mm}\makebox[\linewidth][s]{0 1 2 3 4 5 6 7 8 9 10 } \par

\bigskip
10.	Não consigo prever se vou ter uma noite de sono boa ou ruim. 
\medskip
\hrule
\medskip
\hspace*{6mm}\makebox[\linewidth][s]{0 1 2 3 4 5 6 7 8 9 10 } \par

\bigskip
11.	Tenho pouco controle sobre as consequências negativas de um sono ruim. 
\medskip
\hrule
\medskip
\hspace*{6mm}\makebox[\linewidth][s]{0 1 2 3 4 5 6 7 8 9 10 } \par

\bigskip
12.	Quando me sinto cansado(a), sem energia, ou não funciono bem durante o dia, geralmente é porque não dormi bem na noite anterior. 
\medskip
\hrule
\medskip
\hspace*{6mm}\makebox[\linewidth][s]{0 1 2 3 4 5 6 7 8 9 10 } \par

\bigskip
13.	Acredito que a insônia seja essencialmente o resultado de um desequilíbrio do meu organismo. 
\medskip
\hrule
\medskip
\hspace*{6mm}\makebox[\linewidth][s]{0 1 2 3 4 5 6 7 8 9 10 } \par

\bigskip
14.	Sinto que a insônia está arruinando minha capacidade de aproveitar a vida e me impede de fazer o que eu quero. 
\medskip
\hrule
\medskip
\hspace*{6mm}\makebox[\linewidth][s]{0 1 2 3 4 5 6 7 8 9 10 } \par

\bigskip
15.	Medicação é provavelmente a única solução para a minha falta de sono.  
\medskip
\hrule
\medskip
\hspace*{6mm}\makebox[\linewidth][s]{0 1 2 3 4 5 6 7 8 9 10 } \par

\bigskip
16.	Evito ou cancelo compromissos (sociais, familiares) após uma noite de sono ruim.
\medskip
\hrule
\medskip
\hspace*{6mm}\makebox[\linewidth][s]{0 1 2 3 4 5 6 7 8 9 10 } \par
}

\end{flushleft}

\section{Questionário de Aceitação dos Problemas no Sono (QAPS)}
\label{qaps}

Abaixo você irá encontrar uma lista de afirmações. Por favor, avalie o quanto você concorda com cada frase escolhendo uma alternativa. Responda da melhor forma que puder mesmo que você não tenha (mais) problemas de sono, ou que eles sejam pouco frequentes. Poucas pessoas tem um sono perfeito todas as noites. Deste modo, pense em qualquer dificuldade com sono que tenha, ou já tenha tido, por menor que seja, e responda de acordo.

\begin{table}[h]
\begin{tabular}{ccccccc}
0 & 1 & 2 & 3 & 4 & 5 & 6 \\
Discordo & Concordo & Concordo & Concordo & Concordo & Concordo & Concordo \\
 & muito & levemente & parcialmente & moderadamente & quase & completamente \\
 & pouco &  &  &  & completamente & 
\end{tabular}
\end{table}

\begin{table}[h]
\centering
\begin{tabular}{lllllllll}
1. & Embora as coisas tenham mudado,           & 0 & 1 & 2 & 3 & 4 & 5 & 6 \\
   & estou vivendo uma vida normal             &   &   &   &   &   &   &   \\
   & apesar dos meus problemas de sono.        &   &   &   &   &   &   &   \\
   &                                           &   &   &   &   &   &   &   \\
2. & Eu levo uma vida plena                    & 0 & 1 & 2 & 3 & 4 & 5 & 6 \\
   & apesar de ter problemas de sono.          &   &   &   &   &   &   &   \\
   &                                           &   &   &   &   &   &   &   \\
3. & Minha vida está indo bem                  & 0 & 1 & 2 & 3 & 4 & 5 & 6 \\
   & apesar dos meus problemas de sono         &   &   &   &   &   &   &   \\
   &                                           &   &   &   &   &   &   &   \\
4. & Apesar dos problemas de sono, agora estou & 0 & 1 & 2 & 3 & 4 & 5 & 6 \\
   & seguindo um certo curso na minha vida.    &   &   &   &   &   &   &   \\
   &                                           &   &   &   &   &   &   &   \\
5. & Manter meus problemas de sono sob         & 0 & 1 & 2 & 3 & 4 & 5 & 6 \\
   & controle é minha maior prioridade.        &   &   &   &   &   &   &   \\
   &                                           &   &   &   &   &   &   &   \\
6. & Eu preciso me concentrar em me livrar dos & 0 & 1 & 2 & 3 & 4 & 5 & 6 \\
   & meus problemas de sono.                   &   &   &   &   &   &   &   \\
   &                                           &   &   &   &   &   &   &   \\
7. & É importante eu continuar lutando contra  & 0 & 1 & 2 & 3 & 4 & 5 & 6 \\
   & meus problemas de sono.                   &   &   &   &   &   &   &   \\
   &                                           &   &   &   &   &   &   &   \\
8. & Meus pensamentos e sentimentos sobre      & 0 & 1 & 2 & 3 & 4 & 5 & 6 \\
   & meus problemas de sono devem mudar        &   &   &   &   &   &   &   \\
   & antes de eu dar passos importantes na     &   &   &   &   &   &   &   \\
   & minha vida.                               &   &   &   &   &   &   &  
\end{tabular}
\end{table}

\section{Reconciliation: decisions and documentation form}
\label{appendix:form}
Adapted from: Koller, M., Kantzer, V., Mear, I., Zarzar, K., Martin, M., Greimel, E., ... \& ISOQOL TCA-SIG. (2012). The process of reconciliation: evaluation of guidelines for translating quality-of-life questionnaires. \textit{Expert review of pharmacoeconomics \& outcomes research, 12}(2), 189-197.
\begin{flushleft}
\subsection{Parte I: Decisões}

\textbf{Opções de decisões para a tradução reconciliada}

\smallskip

1.	Tradução A como está\\
2.	Tradução B como está\\
3.	Tradução C como está\\
4.	A com pequenas modificações\\
5.	B com pequenas modificações\\
6.	C com pequenas modificações\\
7.	Mesclar A, B e C como elas são, com A adaptado de B e C\\
8.	Mesclar A, B e C como elas são, com B adaptado de A e B\\
9.	Mesclar A, B e C como elas são, com C adaptado de A e B\\
10.	Mesclar A e B como elas são, com B adaptado de A\\
11.	Mesclar A e B como elas são, com A adaptado de B\\
12.	Mesclar A e C como elas são, com C adaptado de A\\
13.	Mesclar A e C como elas são, com A adaptado de C\\
14.	Mesclar B e C como elas são, com B adaptado de C\\
15.	Mesclar B e C como elas são, com C adaptado de B\\
16.	Mesclar A e B com modificações/adições, com A adaptado de B\\
17.	Mesclar A e B com modificações/adições, com B adaptado de A\\
18.	Preparar uma tradução completamente nova C

\smallskip

\textbf{Critérios de decisão para escolher qualquer uma das opções acima}

\smallskip

\textbf{1.	Fonte e compreensibilidade}

1.1.	 Reflete melhor as definições conceituais e o significado do texto de origem\\
1.2.	 Reflete melhor a ênfase do texto de origem (i.e., qual é o ponto principal do texto de origem)\\
1.3.	 É compreensível para um leigo sem conhecimentos médicos\\
1.4.	 É compreensível para uma população de diversos níveis educacionais\\
1.5.	 É o mais próximo possível do texto de origem\\
1.6.	 É lido com mais naturalidade no idioma de destino\\
\textbf{2.	Cultural}\\
2.1.	 É culturalmente apropriado no âmbito de tópicos sensíveis\\
2.2.	 É culturalmente apropriado no âmbito das diferenças culturais da vida\\
\textbf{3.	Gramatical}\\
3.1.	 A sintaxe está correta\\
3.2.	 As formas e tempos verbais estão corretos\\
3.3.	 Gênero e número estão adaptados e corretos\\
3.4.	 Outros elementos estão corretos (especialmente artigos e preposições)\\
\textbf{4.	Terminologia}\\
4.1.	 Inclui todas as palavras-chave\\
4.2.	 É semanticamente preciso\\
4.3.	 O vocabulário/terminologia é consistente em toda a tradução\\

\subsection{Parte II: Documentação do processo de reconciliação}

Tradução A:\\
Tradução B:\\
Tradução C:\\
Tradução reconciliada:\\
Opção de decisão: escolher uma das opções 1 a 18\\
Critérios de decisão: escolher dentre os critérios 1.1 a 4.3
\end{flushleft}
\end{appendix}